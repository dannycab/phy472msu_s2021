% Options for packages loaded elsewhere
\PassOptionsToPackage{unicode}{hyperref}
\PassOptionsToPackage{hyphens}{url}
%
\documentclass[
]{article}
\usepackage{lmodern}
\usepackage{amssymb,amsmath}
\usepackage{ifxetex,ifluatex}
\ifnum 0\ifxetex 1\fi\ifluatex 1\fi=0 % if pdftex
  \usepackage[T1]{fontenc}
  \usepackage[utf8]{inputenc}
  \usepackage{textcomp} % provide euro and other symbols
\else % if luatex or xetex
  \usepackage{unicode-math}
  \defaultfontfeatures{Scale=MatchLowercase}
  \defaultfontfeatures[\rmfamily]{Ligatures=TeX,Scale=1}
\fi
% Use upquote if available, for straight quotes in verbatim environments
\IfFileExists{upquote.sty}{\usepackage{upquote}}{}
\IfFileExists{microtype.sty}{% use microtype if available
  \usepackage[]{microtype}
  \UseMicrotypeSet[protrusion]{basicmath} % disable protrusion for tt fonts
}{}
\makeatletter
\@ifundefined{KOMAClassName}{% if non-KOMA class
  \IfFileExists{parskip.sty}{%
    \usepackage{parskip}
  }{% else
    \setlength{\parindent}{0pt}
    \setlength{\parskip}{6pt plus 2pt minus 1pt}}
}{% if KOMA class
  \KOMAoptions{parskip=half}}
\makeatother
\usepackage{xcolor}
\IfFileExists{xurl.sty}{\usepackage{xurl}}{} % add URL line breaks if available
\IfFileExists{bookmark.sty}{\usepackage{bookmark}}{\usepackage{hyperref}}
\hypersetup{
  pdftitle={Homework 1 (Due Jan 29th)},
  hidelinks,
  pdfcreator={LaTeX via pandoc}}
\urlstyle{same} % disable monospaced font for URLs
\setlength{\emergencystretch}{3em} % prevent overfull lines
\providecommand{\tightlist}{%
  \setlength{\itemsep}{0pt}\setlength{\parskip}{0pt}}
\setcounter{secnumdepth}{-\maxdimen} % remove section numbering
\usepackage{lmodern}
\usepackage{amssymb,amsmath}
\usepackage{ifxetex,ifluatex}
\usepackage{xcolor}
\usepackage{longtable,booktabs}
% Correct order of tables after \paragraph or \subparagraph
\usepackage{etoolbox}
\usepackage{braket}
\usepackage{graphicx,grffile}
\usepackage[margin=1in]{geometry}
\providecommand{\tightlist}{%
  \setlength{\itemsep}{0pt}\setlength{\parskip}{0pt}}
\setcounter{secnumdepth}{-\maxdimen} % remove section numbering

\title{Homework 1 (Due Jan 29th)}
\author{}
\date{}

\begin{document}
\maketitle

\hypertarget{spin-12-lets-goooo}{%
\subsubsection{1. Spin 1/2; let's goooo}\label{spin-12-lets-goooo}}

Consider a beam of spin-1/2 particles that are sent through a
Stern-Gerlach device. The device measures the z-component of the spin
angular momentum of the particles. After a long time, one quarter
(\(\frac{1}{4}\)) of the particles are observed to be spin up
(\(\ket{+}\)) and three quarters (\(\frac{3}{4}\)) are observed to be
spin down (\(\ket{-}\)).

\begin{enumerate}
\def\labelenumi{\arabic{enumi}.}
\tightlist
\item
  Sketch a histogram of the measured spin values (\(+\hbar/2\);
  \(-\hbar/2\)). See Figs. 1.9, 1.10, or 1.11 in McIntyre for examples.
\item
  What is the expectation value of the z-component of the angular
  momentum (\(\braket{S_z}\)). You should be able to do this using
  probability theory (\(\braket{x} = \sum_i P_i x_i\)). Why does the
  sign of this expectation value make sense?
\item
  In the \(S_z\) basis, the general state vector for any particle in the
  beam is given by \(\ket{\Psi} = a\ket{+} + b\ket{-}\). We have not yet
  determined the coefficients, \(a\) and \(b\). Using the probabilities
  of measuring spin up and spin down for this beam, determine the
  normalized state vector for a particle in the beam, \(\ket{\Psi}\).
  What are we assuming about particles in the beam when we do this?
\item
  Write the normalized state vector from part 3 using the linear algebra
  representation. That is, using
  $\ket{+} \doteq \begin{bmatrix} 1 \cr 0 \end{bmatrix}$ and $\ket{-} \doteq \begin{bmatrix} 0 \cr 1\end{bmatrix}$.
\item
  Using the spin matrix for $S_z \doteq \dfrac{\hbar}{2} \begin{bmatrix} 1 & 0 \cr 0 & -1 \end{bmatrix}$, calculate the expectation value, \(\braket{S_z}\). How does you
  answer compare to part 2?
\item
  Now let's use the $S_x$ spin matrix, $S_x \doteq \dfrac{\hbar}{2}\begin{bmatrix} 0 & 1 \cr 1 & 0 \end{bmatrix}$, to calculate the expectation value, \(\braket{S_x}\). What does
  the sign of this expectation value tell you about the relative
  probabilities of the x-component of the spin angular momentum? That
  is, how do \(P_{+x}\) and \(P_{-x}\) compare?
\item
  Let's check this intuition against the calculated probabilities for
  observing the particles with spin up/down x-components. Calculate the
  probabilities of observing particles in each state:
  \(\|_x\langle + \| \Psi \rangle\|^2\) and
  \(\|_x\langle - \| \Psi \rangle\|^2\). Check that the probabilities
  sum to 1. How do these probabilities compare with your intuition from
  part 6?
\item
  We send the beam through a magnetic field that is directed in
  \(z\)-direction: \(\mathbf{B} = B_0\hat{\mathbf{z}}\). The Hamiltonian
  for that interaction is:
  \(\mathbf{\mu}\cdot \mathbf{B} = \dfrac{qB_0}{m}S_z = \omega_o S_z\)
  where \(\omega_0 = \dfrac{qB_0}{m}\). This Hamiltonian is diagonal in
  the \(S_z\) basis. Write down the energy eigenvalues and eigenstates
  of this Hamiltonian. Why can you simply write down the answer?
\item
  We let the beam time evolve in magnetic field. Using the energy
  eigenvalues, determine the time dependent state vector,
  \(\ket{\Psi(t)}\).
\item
  Determine the probability of observing this time-evolving state vector
  in a spin up/down state for the z-component of the spin angular
  momentum. Is your answer time-dependent? Why or why not?
\end{enumerate}

\hypertarget{the-eigenvalue-problem-the-quantum-crux}{%
\subsubsection{2. The Eigenvalue Problem; The Quantum
Crux}\label{the-eigenvalue-problem-the-quantum-crux}}

Let's investigate a three state quantum system (\(\ket{1}\),
\(\ket{2}\), \(\ket{3}\)). The Hamiltonian for this system is given by:

\[H \doteq \begin{bmatrix} E_1 & 0 & A \cr 0 & E_0 & 0 \cr A & 0 & E_1 \end{bmatrix}\]

In the \(\ket{1} \doteq \begin{bmatrix} 1 \cr 0 \cr 0 \end{bmatrix}\),
\(\ket{2} \doteq \begin{bmatrix} 0 \cr 1 \cr 0 \end{bmatrix}\),
\(\ket{3} \doteq \begin{bmatrix} 0 \cr 0 \cr 1 \end{bmatrix}\) basis the
Hamiltonian is NOT diagonal.

\begin{enumerate}
\def\labelenumi{\arabic{enumi}.}
\tightlist
\item
  Are the state vectors, \(\ket{1}\), \(\ket{2}\), \(\ket{3}\), energy
  eigenstates? How can you tell?
\item
  Diagonalize \(H\) and find the energy eigenvalues. You should find
  three distinct values (\(E_1 - ?\), \(E_0\), and \(E_1 + ?\)). What is
  the value of question mark?
\item
  Sketch an energy level diagram for this system. You can assume
  \(E_0 < E_1\) and \(A < (E_1-E_0)\). What is the ground state, the
  first excited state, the second excited state? How much energy would
  be needed to make the transition between the ground state and the two
  different excited states?
\item
  Now that you have found the energy eigenvalues, use those eigenvalues
  to determine the energy eigenstates in terms of the \(\ket{1}\),
  \(\ket{2}\), \(\ket{3}\) basis. Which eigenstate corresponds to the
  ground state? The first excited state? The second excited state?
\end{enumerate}

\end{document}
