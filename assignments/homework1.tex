% Options for packages loaded elsewhere
\PassOptionsToPackage{unicode}{hyperref}
\PassOptionsToPackage{hyphens}{url}
%
\documentclass[
]{article}
\usepackage{lmodern}
\usepackage{amssymb,amsmath}
\usepackage{ifxetex,ifluatex}
\ifnum 0\ifxetex 1\fi\ifluatex 1\fi=0 % if pdftex
  \usepackage[T1]{fontenc}
  \usepackage[utf8]{inputenc}
  \usepackage{textcomp} % provide euro and other symbols
\else % if luatex or xetex
  \usepackage{unicode-math}
  \defaultfontfeatures{Scale=MatchLowercase}
  \defaultfontfeatures[\rmfamily]{Ligatures=TeX,Scale=1}
\fi
% Use upquote if available, for straight quotes in verbatim environments
\IfFileExists{upquote.sty}{\usepackage{upquote}}{}
\IfFileExists{microtype.sty}{% use microtype if available
  \usepackage[]{microtype}
  \UseMicrotypeSet[protrusion]{basicmath} % disable protrusion for tt fonts
}{}
\makeatletter
\@ifundefined{KOMAClassName}{% if non-KOMA class
  \IfFileExists{parskip.sty}{%
    \usepackage{parskip}
  }{% else
    \setlength{\parindent}{0pt}
    \setlength{\parskip}{6pt plus 2pt minus 1pt}}
}{% if KOMA class
  \KOMAoptions{parskip=half}}
\makeatother
\usepackage{xcolor}
\IfFileExists{xurl.sty}{\usepackage{xurl}}{} % add URL line breaks if available
\IfFileExists{bookmark.sty}{\usepackage{bookmark}}{\usepackage{hyperref}}
\hypersetup{
  pdftitle={Homework 1 (Due Jan 29th)},
  hidelinks,
  pdfcreator={LaTeX via pandoc}}
\urlstyle{same} % disable monospaced font for URLs
\setlength{\emergencystretch}{3em} % prevent overfull lines
\providecommand{\tightlist}{%
  \setlength{\itemsep}{0pt}\setlength{\parskip}{0pt}}
\setcounter{secnumdepth}{-\maxdimen} % remove section numbering
\usepackage{lmodern}
\usepackage{amssymb,amsmath}
\usepackage{ifxetex,ifluatex}
\usepackage{xcolor}
\usepackage{longtable,booktabs}
% Correct order of tables after \paragraph or \subparagraph
\usepackage{etoolbox}
\usepackage{braket}
\usepackage{graphicx,grffile}
\usepackage[margin=1in]{geometry}
\providecommand{\tightlist}{%
  \setlength{\itemsep}{0pt}\setlength{\parskip}{0pt}}
\setcounter{secnumdepth}{-\maxdimen} % remove section numbering

\title{Homework 1 (Due Jan 29th)}
\author{}
\date{}

\begin{document}
\maketitle

This first homework set is meant to remind you of the main concepts from
Quantum 1. It focuses on spin, operators, diagonalization and the
formalism of kets and linear algebra. We will build on all of these
tools and ideas this semester, so you should make sure you have these
ideas down before next week. In addition, there's a couple computational
problem that are meant to introduce you to the linear algebra package
for Python.

\hypertarget{spin-12-lets-goooo}{%
\subsubsection{1. Spin 1/2; let's goooo}\label{spin-12-lets-goooo}}

Consider a beam of spin-1/2 particles that are sent through a
Stern-Gerlach device. The device measures the z-component of the spin
angular momentum of the particles. After a long time, one quarter
(\(\frac{1}{4}\)) of the particles are observed to be spin up
(\(\ket{+}\)) and three quarters (\(\frac{3}{4}\)) are observed to be
spin down (\(\ket{-}\)).

\begin{enumerate}
\def\labelenumi{\arabic{enumi}.}
\tightlist
\item
  Sketch a histogram of the measured spin values (\(+\hbar/2\);
  \(-\hbar/2\)). See Figs. 1.9, 1.10, or 1.11 in McIntyre for examples.
\item
  What is the expectation value of the z-component of the angular
  momentum (\(\braket{S_z}\)). You should be able to do this using
  probability theory (\(\braket{x} = \sum_i P_i x_i\)). Why does the
  sign of this expectation value make sense?
\item
  In the \(S_z\) basis, the general state vector for any particle in the
  beam is given by \(\ket{\Psi} = a\ket{+} + b\ket{-}\). We have not yet
  determined the coefficients, \(a\) and \(b\). Using the probabilities
  of measuring spin up and spin down for this beam, determine the
  normalized state vector for a particle in the beam, \(\ket{\Psi}\).
  What are we assuming about particles in the beam when we do this?
\item
  Write the normalized state vector from part 3 using the linear algebra
  representation. That is, using \(\ket{+} \doteq \begin{bmatrix} 1 \cr 0 \end{bmatrix}\) and \(\ket{-} \doteq \begin{bmatrix} 0 \cr 1\end{bmatrix}\).
\item
  Using the spin matrix for \(S\_z \doteq \dfrac{\hbar}{2}\begin{bmatrix} 1 & 0 \cr 0 & -1 \end{bmatrix}\), calculate the expectation value, \(\braket{S_z}\). How does you
  answer compare to part 2?
\item
  Now let's use the \(S_x\) spin matrix, \(S\_x \doteq \dfrac{\hbar}{2}\begin{bmatrix} 0 & 1 \cr 1 & 0 \end{bmatrix}\), to calculate the expectation value, \(\braket{S_x}\). What does
  the sign of this expectation value tell you about the relative
  probabilities of the x-component of the spin angular momentum? That
  is, how do \(P_{+x}\) and \(P_{-x}\) compare?
\item
  Let's check this intuition against the calculated probabilities for
  observing the particles with spin up/down x-components. Calculate the
  probabilities of observing particles in each state:
  \(\|_x\langle + \| \Psi \rangle\|^2\) and
  \(\|_x\langle - \| \Psi \rangle\|^2\). Check that the probabilities
  sum to 1. How do these probabilities compare with your intuition from
  part 6?
\item
  We send the beam through a magnetic field that is directed in
  \(z\)-direction: \(\mathbf{B} = B_0\hat{\mathbf{z}}\). The Hamiltonian
  for that interaction is:
  \(\mathbf{\mu}\cdot \mathbf{B} = \dfrac{qB_0}{m}S_z = \omega_o S_z\)
  where \(\omega_0 = \dfrac{qB_0}{m}\). This Hamiltonian is diagonal in
  the \(S_z\) basis. Write down the energy eigenvalues and eigenstates
  of this Hamiltonian. Why can you simply write down the answer?
\item
  We let the beam time evolve in magnetic field. Using the energy
  eigenvalues, determine the time dependent state vector,
  \(\ket{\Psi(t)}\).
\item
  Determine the probability of observing this time-evolving state vector
  in a spin up/down state for the z-component of the spin angular
  momentum. Is your answer time-dependent? Why or why not?
\end{enumerate}

\hypertarget{the-eigenvalue-problem-the-quantum-crux}{%
\subsubsection{2. The Eigenvalue Problem; The Quantum
Crux}\label{the-eigenvalue-problem-the-quantum-crux}}

Let's investigate a three state quantum system (\(\ket{1}\),
\(\ket{2}\), \(\ket{3}\)). The Hamiltonian for this system is given by:

\[H \doteq \begin{bmatrix} E_1 & 0 & A \cr 0 & E_0 & 0 \cr A & 0 & E_1 \end{bmatrix}\]

In the \(\ket{1} \doteq \begin{bmatrix} 1 \cr 0 \cr 0 \end{bmatrix}\),
\(\ket{2} \doteq \begin{bmatrix} 0 \cr 1 \cr 0 \end{bmatrix}\),
\(\ket{3} \doteq \begin{bmatrix} 0 \cr 0 \cr 1 \end{bmatrix}\) basis the
Hamiltonian is NOT diagonal.

\begin{enumerate}
\def\labelenumi{\arabic{enumi}.}
\tightlist
\item
  Are the state vectors, \(\ket{1}\), \(\ket{2}\), \(\ket{3}\), energy
  eigenstates? How can you tell?
\item
  Diagonalize \(H\) and find the energy eigenvalues. You should find
  three distinct values (\(E_1 - ?\), \(E_0\), and \(E_1 + ?\)). What is
  the value of question mark?
\item
  Sketch an energy level diagram for this system. You can assume
  \(E_0 < E_1\) and \(A < (E_1-E_0)\). What is the ground state, the
  first excited state, the second excited state? How much energy would
  be needed to make the transition between the ground state and the two
  different excited states?
\item
  Now that you have found the energy eigenvalues, use those eigenvalues
  to determine the energy eigenstates in terms of the \(\ket{1}\),
  \(\ket{2}\), \(\ket{3}\) basis. Which eigenstate corresponds to the
  ground state? The first excited state? The second excited state?
\end{enumerate}

\hypertarget{spin-games}{%
\subsubsection{3. Spin Games}\label{spin-games}}

Diagonalization is one of the key processes for quantum mechanics; it is
needed to find the eigenvalues and eigenvectors of operators. Two by two
matrices are relatively easy to diagonalize analytically as you
typically solve quadratic questions. Higher order operators are more
challenging to diagonalize because with each new dimension you introduce
an additional term in the polynomial you attempt to solve. In this
problem, you will diagonalize the \(S_x\) operator for a spin 1 system.

\[S_x \doteq \dfrac{\hbar}{\sqrt{2}}\begin{bmatrix} 0 & 1 & 0 \cr 1 & 0 & 1 \cr 0 & 1 & 0 \end{bmatrix}\]

\begin{enumerate}
\def\labelenumi{\arabic{enumi}.}
\tightlist
\item
  Diagonalize \(S_x\) to find the eigenvalues of the operator. What is
  the order of the polynomial you have to solve? Why does that make
  sense?
\item
  Using these eigenvalues, find the eigenvectors of the \(S_x\)
  operator.
\item
  Can you measure \(S_x\) and \(S_z\) at the same time? Why or why not?
\item
  Compute the commutator \([S_x,S_z]\). What does that result tell you
  about the uncertainty principle as it relates to a spin-1 system?
\end{enumerate}

\hypertarget{time-evolution}{%
\subsubsection{4. Time Evolution}\label{time-evolution}}

General time evolution of states is hard, but when the Hamiltonian is
time independent and we know the energy eigenstates, time evolution is
pretty much a procedural calculation (just multiply by
\(e^\frac{-i E_n t}{\hbar}\)). However, we have to have the appropriate
energy eigenstates, which can be tricky. Let's explore how we get there
with a system where the energy eigenstates are not as straightforward.

Consider a spin-1/2 particle that is allowed to evolve in a uniform
magnetic field. The particle is initially in the state
\(\ket{\psi(0)} = \ket{+}_n\) where
\(\hat{n} = (\hat{x}+\hat{y})/\sqrt{2}\). The magnetic field that the
particle is placed in is given by
\(\mathbf{B} = B_0 (\hat{x}+\hat{z})/\sqrt{2}\).

\begin{enumerate}
\def\labelenumi{\arabic{enumi}.}
\tightlist
\item
  In this situation, the state vector is not an energy eigenvector; the
  state vector is not aligned to the magnetic field. We first need to
  write \(\ket{\psi(0)}\) in the usual \(\ket{+}, \ket{-}\) basis. In
  general,
  \(\ket{n}_+ = \cos\frac{\theta}{2}\ket{+} + e^{i\phi}\sin\frac{\theta}{2}\ket{-}\).
  Sketch the initial state vector in 3D space and determine \(\theta\)
  and \(\phi\). Write down \(\ket{\psi(0)}\) in the usual
  \(\ket{+}, \ket{-}\) basis.
\item
  As the particle has a magnetic moment, it will tend to align with the
  magnetic field, which does not point in the direction of the initial
  state vector. So, we also need to determine the initial state in terms
  of the basis of the magnetic field direction. This is the energy basis
  for this problem, so we can use our simple expansion with exponential
  terms (with \(\pm \hbar \omega/2\)). (Recall: finding the state in the
  energy basis of the problem makes it easy for us to use this
  expansion.). In general, the energy basis states are given by:
  \[\ket{+}_{n'} = \cos\frac{\theta}{2}\ket{+} + e^{i\phi}\sin\frac{\theta}{2}\ket{-}\]
  and
  \[\ket{-}_{n'} = \cos\frac{\theta}{2}\ket{+} - e^{i\phi}\sin\frac{\theta}{2}\ket{-}\]
  where \(\hat{n}' = (\hat{x}+\hat{z})/\sqrt{2}\). Sketch the magnetic
  field field vector in 3D space and determine \(\theta\) and \(\phi\).
  Write down \(\ket{\psi(0)}\) in the \(\ket{+}_{n'}, \ket{-}_{n'}\)
  basis; you will need to use projection operators and/or the closure
  relationship.
\item
  Now that you have \(\ket{\psi(0)}\) in the energy
  (\(\ket{+}_{n'}, \ket{-}_{n'}\)) basis, use the exponential expansion
  to find \(\ket{\psi(t)}\).
\item
  Finally, compute the time dependent probability of measuring
  \(S_y = +\hbar/2\). You should find that it is proportional to:
  \(2 + \sqrt{2}\cos\omega t + \sin \omega t\). Given that it is time
  dependent, what does that tell you about the state (re: stationary
  states)?
\end{enumerate}

\hypertarget{jupyter-and-linear-algebra}{%
\subsubsection{5. Jupyter and Linear
Algebra}\label{jupyter-and-linear-algebra}}

\textbf{You will turn in this question using a
\href{https://www.dropbox.com/request/u3J2phx9zlkCZXfpzJaR}{Dropbox file
request}. Turn in the notebook, not a PDF of it.}

Python is a powerful and flexible programming language that can be used
to solve many kinds of scientific problems. The wide variety of modules
and libraries that are available for Python have specific uses for
particular kinds of problems. In this problem, you will learn to use the
\texttt{numpy} module and the associated \texttt{linalg} library to do
common linear algebra manipulations that readily appear in quantum
mechanics problems. In so doing, you will solve homework problem 1
again, but this time using Python. The idea is that while most of the
homework problems you work are analytically tractable and not terribly
cumbersome, other problems you might encounter will not be and using
something like \texttt{numpy.linalg} will be very useful. I recommend
using \href{https://www.anaconda.com/products/individual}{Anaconda
Python} as it has all the libraries and modules we need.

For this assignment, download this
\href{./notebooks/Homework1_Problem5_STUDENT.ipynb}{Jupyter notebook}
and work through the notebook. All the instructions appear in the
notebook. The design is such that you are shown how to do some
calculation, and then asked to translate that calculation to the problem
at hand.

\end{document}
