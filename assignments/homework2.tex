% Options for packages loaded elsewhere
\PassOptionsToPackage{unicode}{hyperref}
\PassOptionsToPackage{hyphens}{url}
%
\documentclass[
]{article}
\usepackage{lmodern}
\usepackage{amssymb,amsmath}
\usepackage{ifxetex,ifluatex}
\ifnum 0\ifxetex 1\fi\ifluatex 1\fi=0 % if pdftex
  \usepackage[T1]{fontenc}
  \usepackage[utf8]{inputenc}
  \usepackage{textcomp} % provide euro and other symbols
\else % if luatex or xetex
  \usepackage{unicode-math}
  \defaultfontfeatures{Scale=MatchLowercase}
  \defaultfontfeatures[\rmfamily]{Ligatures=TeX,Scale=1}
\fi
% Use upquote if available, for straight quotes in verbatim environments
\IfFileExists{upquote.sty}{\usepackage{upquote}}{}
\IfFileExists{microtype.sty}{% use microtype if available
  \usepackage[]{microtype}
  \UseMicrotypeSet[protrusion]{basicmath} % disable protrusion for tt fonts
}{}
\makeatletter
\@ifundefined{KOMAClassName}{% if non-KOMA class
  \IfFileExists{parskip.sty}{%
    \usepackage{parskip}
  }{% else
    \setlength{\parindent}{0pt}
    \setlength{\parskip}{6pt plus 2pt minus 1pt}}
}{% if KOMA class
  \KOMAoptions{parskip=half}}
\makeatother
\usepackage{xcolor}
\IfFileExists{xurl.sty}{\usepackage{xurl}}{} % add URL line breaks if available
\IfFileExists{bookmark.sty}{\usepackage{bookmark}}{\usepackage{hyperref}}
\hypersetup{
  pdftitle={Homework 2 (Due Feb 5th)},
  hidelinks,
  pdfcreator={LaTeX via pandoc}}
\urlstyle{same} % disable monospaced font for URLs
\setlength{\emergencystretch}{3em} % prevent overfull lines
\providecommand{\tightlist}{%
  \setlength{\itemsep}{0pt}\setlength{\parskip}{0pt}}
\setcounter{secnumdepth}{-\maxdimen} % remove section numbering
\usepackage{lmodern}
\usepackage{amssymb,amsmath}
\usepackage{ifxetex,ifluatex}
\usepackage{xcolor}
\usepackage{longtable,booktabs}
% Correct order of tables after \paragraph or \subparagraph
\usepackage{etoolbox}
\usepackage{braket}
\usepackage{graphicx,grffile}
\usepackage[margin=1in]{geometry}
\providecommand{\tightlist}{%
  \setlength{\itemsep}{0pt}\setlength{\parskip}{0pt}}
\setcounter{secnumdepth}{-\maxdimen} % remove section numbering

\title{Homework 2 (Due Feb 5th)}
\author{}
\date{}

\begin{document}
\maketitle

Homework 2 focuses on the position representation of the state vector
and the infinite square well. It is a little shorter because we have a
take home quiz this week.

\hypertarget{symmetry-and-the-infinite-square-well}{%
\subsection{1. Symmetry and the Infinite Square
Well}\label{symmetry-and-the-infinite-square-well}}

McIntyre solves the infinite square well problem where the boundaries of
the well are 0 and \(L\). As a result, the energy eigenvalues are
\(E_n = \dfrac{n^2\pi^2\hbar^2}{2mL^2}\) for \(n = 1,2,3,\dots\) and the
energy eigenstates, in the position representation, are
\(\phi_n(x) = \sqrt{\dfrac{2}{L}} \sin \dfrac{n\pi x}{L}\) for
\(n = 1,2,3,\dots\) again.

\begin{enumerate}
\def\labelenumi{\arabic{enumi}.}
\tightlist
\item
  Consider changing the boundaries from {[}0, \(L\){]} to {[}\(-L\),
  \(L\){]}. The potential remains the same, but the well width has been
  expanded. Without performing a calculation, determine the new energy
  eigenvalues. Explain how you determined them and discuss whether the
  states are more or less energetic than the original system.
\item
  Given the change in the symmetry of the well, what functional
  dependence do you now expect for the position representation of the
  state vector? That is, does a pure \(\sin\) function still work? Based
  on what we did in class, what form should \(\phi_n\) take now? Explain
  why this choice makes sense.
\item
  Write down the full form of the position representation of the state
  vector. That is, what precisely is \(\phi_n(x)\)? Leave no
  undetermined coefficients and make sure it is normalized. You should
  not have to solve the infinite square well problem to determine this,
  but use symmetry arguments and boundary conditions.
\item
  Check that you answer to part 1 and part 3 make sense by using the
  eigenvalue equation
  \(-\dfrac{\hbar^2}{2m} \dfrac{d^2\phi_n(x)}{dx^2} = E_n \phi_n(x)\) to
  check your results.
\end{enumerate}

\hypertarget{expectation-values}{%
\subsection{2. Expectation Values}\label{expectation-values}}

One of the things that is really frustrating about quantum mechanics is
determining what physical meaning different things have. For example, if
I were asked: what is the physical meaning of the wavefunction? I would
likely answer: ¯\textbackslash\_(ツ)\_/¯ Because the physical meaning in
quantum mechanics stems from observables. Things like: energy, position,
momentum, and probabilities or expectation values of making
measurements. So, while the wavefunction is incredibly useful, its
physical meaning comes from using it to determine observables.

Let's go back to the form of the wavefunction determined in McIntyre
(with well boundaries, \([0,L]\) ):
\(\phi_n(x) = \sqrt{\dfrac{2}{L}} \sin \dfrac{n\pi x}{L}\) for
\(n = 1,2,3,\dots\)

\begin{enumerate}
\def\labelenumi{\arabic{enumi}.}
\tightlist
\item
  Sketch the first 3 states. If you were asked to determine the
  expectation value of the position, \(\langle x \rangle\), for each of
  these states given your sketches, what would you conclude? Explain how
  you determined your answer.
\item
  Compute \(\langle x \rangle\) for all states. Comment on your result.
\item
  Can you use the same arguments from part 1 to determine
  \(\langle p_x \rangle\)? Why or why not?
\item
  Compute \(\langle p_x \rangle\) for all states. Why does this average
  value for \(p_x\) make sense?
\end{enumerate}

\hypertarget{time-evolution-of-a-superposition-state}{%
\subsection{3. Time Evolution of a Superposition
State}\label{time-evolution-of-a-superposition-state}}

As we determined earlier, energy eigenstates are particularly useful
when we want to see how the system evolves in time. As the infinite
square well Hamiltonian is time independent, we expect that the energy
eigenstates are stationary states. Let's bring this thinking to the
position representation of the energy eigenstates.

Consider the superposition state
\(\ket{\psi(t=0)} = A \left(\ket{\phi_1} - \dfrac{1}{\sqrt{2}}\ket{\phi_2}\right)\)
where \(\ket{\phi_n}\) are energy eigenstates of the infinite square
well.

\begin{enumerate}
\def\labelenumi{\arabic{enumi}.}
\tightlist
\item
  Make sure the state is normalized. Write down the normalized state
  using ket notation.
\item
  In which state is the system more likely to be found? With what
  probability do you expect to find each state?
\item
  At \(t=0\) compute the expectation value of the Hamiltonian,
  \(\langle \hat{H} \rangle\). Do this first using the ket
  representation, then using the full integral formalism. Comment on the
  relative difficulty of the two approaches and which one makes more
  sense to use in this case.
\item
  How does \(\langle \hat{H} \rangle\) from part 3 fit with your
  discussion in part 2?
\item
  Write down the time evolution of the state in both the ket and
  wavefunction representation.
\item
  Using the representation of your choice (recall part 3), compute
  \(\langle \hat{H} \rangle\) for the time evolving state. How does your
  result compare to part 3? Why would this result make sense?
\end{enumerate}

\hypertarget{using-python-to-compute-eigenvalues-and-eigenvectors}{%
\subsection{4. Using Python to compute Eigenvalues and
Eigenvectors}\label{using-python-to-compute-eigenvalues-and-eigenvectors}}

\textbf{You will turn in this question using a
\href{https://www.dropbox.com/request/vscBnPjYqOEUk1VyPOO0}{Dropbox file
request}. Turn in the notebook, not a PDF of it.}

For this question, download this
\href{./notebooks/Homework2_Problem4_STUDENT.ipynb}{Jupyter notebook}
and work through the notebook. All the instructions appear in the
notebook. The design is such that you are shown how to do some
calculation, and then asked to translate that calculation to the problem
at hand.

\end{document}
